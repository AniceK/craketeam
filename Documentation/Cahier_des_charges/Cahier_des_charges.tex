%%%%%%%%%%%%%%%%%%%%%%%%%%%%%%%%%%%%%%%%%
% Programming/Coding Assignment
% LaTeX Template
%
% This template has been downloaded from:
% http://www.latextemplates.com
%
% Original author:
% Ted Pavlic (http://www.tedpavlic.com)
%
% Note:
% The \lipsum[#] commands throughout this template generate dummy text
% to fill the template out. These commands should all be removed when 
% writing assignment content.
%
% This template uses a Perl script as an example snippet of code, most other
% languages are also usable. Configure them in the "CODE INCLUSION 
% CONFIGURATION" section.
%
%%%%%%%%%%%%%%%%%%%%%%%%%%%%%%%%%%%%%%%%%

%----------------------------------------------------------------------------------------
%	PACKAGES AND OTHER DOCUMENT CONFIGURATIONS
%----------------------------------------------------------------------------------------

\documentclass[11pt]{article}

\usepackage{fancyhdr} % Required for custom headers
\usepackage{lastpage} % Required to determine the last page for the footer
\usepackage{extramarks} % Required for headers and footers
\usepackage[usenames,dvipsnames]{color} % Required for custom colors
\usepackage{graphicx} % Required to insert images
\usepackage{listings} % Required for insertion of code
\usepackage{courier} % Required for the courier font
\usepackage{lipsum} % Used for inserting dummy 'Lorem ipsum' text into the template
\usepackage{multirow} %Ligne multiple pour les tableaux
\usepackage[utf8]{inputenc}
\usepackage{indentfirst} %Indentation début de paragraphe

\usepackage{colortbl} %Clouleur tableau protoypes de fonctions


% Margins
\topmargin=-0.45in
\evensidemargin=0in
\oddsidemargin=0in
\textwidth=6.5in
\textheight=9.0in
\headsep=0.25in

\linespread{1.1} % Line spacing

% Set up the header and footer
\pagestyle{fancy}
\lhead{\hmwkAuthorName} % Top left header
\chead{\hmwkClass\ - \hmwkTitle} % Top center head
\rhead{\firstxmark} % Top right header
\lfoot{\lastxmark} % Bottom left footer
\cfoot{} % Bottom center footer
\rfoot{Page\ \thepage\ sur\ \protect\pageref{LastPage}} % Bottom right footer
\renewcommand\headrulewidth{0.4pt} % Size of the header rule
\renewcommand\footrulewidth{0.4pt} % Size of the footer rule

\setlength\parindent{10pt} % Removes all indentation from paragraphs




%----------------------------------------------------------------------------------------
%	NAME AND CLASS SECTION
%----------------------------------------------------------------------------------------

\newcommand{\hmwkTitle}{PROJET JEU} % Titre du document
\newcommand{\hmwkDueDate}{Mercredi 5 février 2014} % Date
\newcommand{\hmwkClass}{CAHIER DES CHARGES } % Type de document
\newcommand{\hmwkClassInstructor}{ } % Teacher/lecturer
\newcommand{\hmwkAuthorName}{Groupe A} % Your name
\newcommand{\hmwkAuthorClasse}{L3 info SPI} % Classe


%----------------------------------------------------------------------------------------
%	TITLE PAGE
%----------------------------------------------------------------------------------------

\title{
\pagenumbering{roman} \setcounter{page}{0} %La page courante sera numérotée en roman et aura l'indice 0 => Pas de numéro car pas de 0 en roman
\vspace{2in}
\textmd{\textbf{\hmwkClass:\ \hmwkTitle}}\\
\normalsize\vspace{0.1in}\small{\hmwkDueDate}\\
\vspace{0.1in}\large{\textit{\hmwkClassInstructor\ }}
\vspace{3in}
}

\author{\textbf{\hmwkAuthorName}}


\date{\hmwkAuthorClasse} % Insert date here if you want it to appear below your name

%----------------------------------------------------------------------------------------

\begin{document}

\thispagestyle{empty}
\maketitle
\newpage


%----------------------------------------------------------------------------------------
%	TABLE OF CONTENTS
%----------------------------------------------------------------------------------------

%\setcounter{tocdepth}{1} % Uncomment this line if you don't want subsections listed in the ToC

\thispagestyle{empty}
\pagenumbering{arabic} \setcounter{page}{0} %Le reste du document est numéroté en arabic à partir de la page 1
\renewcommand\contentsname{Sommaire}
\tableofcontents
\newpage


%----------------------------------------------------------------------------------------
%	Présentation du projet
%----------------------------------------------------------------------------------------

\section{Présentation du projet}

\subsection{Contexte}

Lors du second semestre de troisième année de licence SPI, il est demandé de réaliser un projet. Ce projet à thème imposé n'est autre que le développement d'un logiciel complexe demandant un travail en équipe.\\
Notre équipe de cinq étudiants aura donc du 24 janvier au 16 Mai 2014 pour mettre en œuvre les préceptes de Génie Logiciel vus au premier semestre afin de présenter un logiciel potentiellement commercialisable au client, ici représenté par les enseignants.

\subsection{Objectifs}

L'objectif de ce projet est la conception d'une application permettant la création et l'aide à la résolution de puzzles de type picross (henjie).\\
Le joueur pourra commencer des puzzles de différentes tailles ; le but étant de terminer le plus vite possible la partie.


\subsection{Utilisateur cible}

Ce logiciel n'est destiné qu'à un seul joueur à la fois. Cependant, il permet à de nombreux joueurs d'apposer leur score à la table des scores.


\subsection{État de l'art}

Le jeu permet différentes fonctionnalités :
\begin{itemize}
   \item La création et édition d'une grille de picross,
   \item La résolution du puzzle avec ou sans aide à la résolution,
   \item La sauvegarde et le chargement d'une partie,
   \item L'enregistrement et consultation des scores,
   \item Le partage de ses grilles déjà jouées.
\end{itemize}



%----------------------------------------------------------------------------------------
%	Contraintes initiales
%----------------------------------------------------------------------------------------

\section{Contraintes initiales}


\subsection{Contraintes de jeu}


\begin{itemize}
   \item Le jeu est un picross, le joueur doit noircir les cases en fonction des information donné par les chiffres en
       tête de ligne et de colonne,
   \item Le jeu doit proposer plusieurs tailles de grille (5x5 10x10 15x15 25x25),
   \item L'application doit proposer un éditeur de grille.
\end{itemize}




\subsection{Contraintes de conception}


\begin{itemize}
   \item Le langage de programmation est Ruby/GTK,
   \item Les paramètres de l’application doivent être externalisés afin de pouvoir être modifiés par le joueur (taille
       de la grille),
   \item Le logiciel doit être développé en Programmation Orientée Objet (POO),
   \item Le jeu doit supporter plusieurs langues,
   \item Le joueur doit pouvoir sauvegarder une partie pour la reprendre ultérieurement.
\end{itemize}




\subsection{Contraintes temporelles}


\begin{itemize}
    \item Le client impose une réunion hebdomadaire entre les différents acteurs de notre groupe,
    \item Le projet final doit être présenté au client le 16 mai 2014.
\end{itemize}




\subsection{Contraintes matérielles}


\begin{itemize}
    \item Le présent Cahier des Charges doit être remis au client le 16 mai 2014,
    \item Au terme de chaque réunion un compte rendu doit être remis au client.
\end{itemize}


%----------------------------------------------------------------------------------------
%	Spécifications du projet
%----------------------------------------------------------------------------------------


\section{Spécification du projet}

\subsection{Règles détaillées}

\subsubsection{Éléments constituant le jeu}

    \begin{description}
        \item[Grille] : La grille est constituée d'un carré dont le coté est un multiple de 5 cases. Il y a ainsi cinq tailles de grilles différentes, 5x5, 10x10, 15x15, 20x20 et 25x25.
        \item[Case] : Les cases peuvent être noircies, en cliquant dessus à l'aide de la souris. Il est possible de décocher la case en cliquant dessus à nouveau. Il est également possible de cliquer sur une case à l'aide de la souris pour indiquer que la case restera vide, qu'elle ne sera pas noircie.
        \item[Indications extérieures] : Les indications extérieures sont des suites de chiffres sur les bords gauche et supérieur, indiquant le nombre de case à noircir, respectivement dans les lignes et dans les colonnes.
        \item[Chronomètre] : le chronomètre indique le temps qui s'est
    \end{description}

\subsubsection{Principe de fonctionnement du jeu}

\paragraph{But du jeu}
Le but du jeu est de noircir toutes les cases devant être noircies dans le plus court laps de temps possible.

\paragraph{Début d'une partie}

Au début d'une partie, la grille est vierge, et le chronomètre
\paragraph{Le jeu au tour par tour}

Le déroulement du jeu se fait tour par tour. Une action pour un tour correspond à un déplacement (dans les quatre directions nord, sud, est et ouest sauf cas de cases inaccessibles) plus une interaction dans la case où se trouve le joueur à la fin de ce déplacement : il n'est possible d’interagir qu'avec les éléments présents sur la même case que le personnage. Lors d'un tour, le joueur peut décider de rester sur place (ce qui compte pour un déplacement nul). Il pourra ensuite agir sur la case sur laquelle il est resté avant la fin du tour. Une fois le tour du joueur fini, le monde évolue (mouvement des ennemis, génération de nouveaux ennemis et/ou objets sur la carte).\\
Se reposer (utilisation d'un repos) permet au joueur de récupérer 10\% de la totalité de son énergie à chaque tour pendant 10 tours sauf si ce dernier est attaqué durant son someil, auquel cas le repos s'arrêtre et un combat s'engage.\\
Lors d'une interaction avec un PNJ d'aide, chaque transaction compte pour un tour~: pour acheter trois objets, il faudra rester sur place trois fois de suite. Le joueur ne se déplaçant pas, il ne consomme (ni ne gagne) aucune énergie.\\
Dans le cas où plusieurs éléments sont présents sur une même case, le joueur peut décider d’interagir avec, dans l'ordre qu'il souhaite, sachant que chaque action équivaut à un tour en restant sur place (ramassage de plusieurs objets, interaction avec un/des PNJ d'aide). Si des ennemis se trouvent parmi ces éléments, la règle est différente~: les ennemis doivent être traités en priorité et en un seul tour, ce qui signifie que l'énergie soustraite à celle du joueur correspond à la somme des énergies de tous les ennemis présents sur la même case et qu'il est impossible au joueur d'utiliser des équipements ou de la nourriture entre chacun des combats (seulement pour le premier). Si le joueur survit, il peut ensuite interagir avec les éléments non offensifs restants (s'il y en a) comme vu précédemment.

\paragraph{Les niveaux}

Lorsque le joueur remporte un combat il acquiert de l’expérience. Elle lui permet de devenir plus puissant afin de combattre des ennemis qui étaient auparavant hors de portée du joueur.
Le joueur se voit attribuer un bonus de 1,2 fois son énergie, ainsi qu'un repos tout les cinq niveaux.\\
À chaque passage de niveau, des ennemis et des objets supplémentaires apparaissent, leur nombre dépendant de la difficulté. Le niveau des ennemis ajoutés est calculé en fonction de celui du joueur.


\paragraph{Lancement d'une partie}

Au lancement d'une partie, on propose au joueur d'entrer son pseudonyme et de choisir une difficulté entre novice, moyen et expert. C'est la difficulté d'une partie qui détermine les proportions des différents éléments du jeu~:

\begin{itemize}
 \item \textbf{Novice} :

	\begin{itemize}
		\item Quantités PNJ :
		
		\begin{itemize}
			\item Ennemis : 50 au départ et 5 à chaque nouvelle génération~;
			\item Aides : 50~;
		\end{itemize}

		\item Pourcentage Terrain : 80 \% de la dépense énergétique du terrain~;
		\item Quantité d'objet : 100 répartis dans toute la carte~;
		\item Dimensions carte : 500 cases * 500 cases~;
		\item Energie initiale du joueur : 1000~;
		\item Nombre de repos initial du joueur : 15~;
		\item Nombre de tours entre chaque génération d'ennemis : 15.

	\end{itemize}

 \item \textbf{Moyen} :

		\begin{itemize}
			\item Quantités de PNJ :
			
			\begin{itemize}
				\item Ennemis : 100 au départ et 10 à chaque nouvelle génération~;
				\item Aides : 25~;
			\end{itemize}

			\item Pourcentage Terrain : 100 \% de la dépense énergétique du terrain~;
			\item Quantité d'objet : 75 répartis dans toute la carte~;
			\item Dimensions carte : 1000 cases * 1000 cases~;
			\item Energie initiale du joueur : 750~;
			\item Nombre de repos initial du joueur : 10~;
			\item Nombre de tours entre chaque génération d'ennemis : 10.

		\end{itemize}
                     \item \textbf{Expert} :
		\begin{itemize}
			\item Quantités de PNJ :
			
			\begin{itemize}
				\item Ennemis : 200 au départ et 200 à chaque nouvelle génération~;
				\item Aides : 25~;
			\end{itemize}

			\item Pourcentage Terrain : 120 \% de la dépense énergétique du terrain~;
			\item Quantité d'objet : 50 répartis dans toute la carte~;
			\item Dimensions carte : 2000 cases * 2000 cases~;
			\item Energie initiale du joueur : 500~;
			\item Nombre de repos initial du joueur : 5~;
			\item Nombre de tours entre chaque génération d'ennemis : 5.

		\end{itemize}
\end{itemize}

Ensuite une nouvelle partie commence.


\paragraph{Le menu en cours de partie}

En cours de partie, le joueur peut accéder à un menu spécial où se trouvent~:


\begin{itemize}
   \item \textbf{Sauvegarde de la partie : }L'utilisateur peut sauvegarder sa partie en cours. Le joueur peut ensuite continuer à jouer ou quitter le jeu~;
   \item \textbf{Son : }Active ou désactive le son~;
   \item \textbf{Quitter la partie : }Renvoie le joueur au menu principal. Il devra avoir pris le soin de sauvegarder ou non sa partie.  
\end{itemize}


\paragraph{L'interface joueur}

Le joueur peut exécuter toutes les actions via l'interface graphique, il pourra également utiliser le pavé directionnel du clavier pour se déplacer et la touche échap pour accéder au menu. Un bouton lui permet d’accéder à l'inventaire. Une console affiche les interactions effectués, objets trouvés et utilisés. Il voit également une barre représentant son expérience et une barre représentant son énergie, avec leur valeur affichée au dessus, ainsi que les objets dont il est équipé, les repos dont il dispose et l'or de sa cagnotte.
D'autre part si le même méchant est présent plusieurs fois sur la même case, un nombre apparaît à côté de son image pour spécifier le nombre.






\subsection{Besoins fonctionnels}

\vspace{0.5cm}

\small
\begin{tabular}{|p{4cm}|p{10cm}|p{1cm}|}
  \hline
  \textbf{FONCTIONNALITE} & \textbf{DESCRIPTION} & \textbf{M/m*}\\
  \hline
     \multirow{8}{*}{Gestion du jeu}
        & Lancer une nouvelle partie                                  & M \\  \cline{2-3}
        & Sauvegarder une partie (à tout instant)                     & M \\  \cline{2-3}
        & Charger une partie sauvegardée                              & M \\  \cline{2-3}
        & Supprimer une partie sauvegardée                            & M \\  \cline{2-3}
        & Quitter le jeu                                              & M \\  \cline{2-3}
        & Choisir lea taille de la grille (Facile, Moyen, Difficile)  & M \\  \cline{2-3}
        & Choisir la langue                                           & m \\  \cline{2-3}
        & Afficher les statistiques/classement (nom du joueur, nombre de partie jouées, victoire, temps, score) & M \\  \cline{2-3}
        & Afficher une aide                                           & M \\  \cline{2-3}
        & Activer / Désactiver le son                                 & m \\
  \hline
     \multirow{2}{*}{Gestion du joueur}
        & Choisir un pseudo pour le joueur                                            & m \\  \cline{2-3}           & Gerer ses parties 														  & M \\  \cline{2-3}
  \hline
     \multirow{3}{*}{Interface du jeu}
        & Afficher les informations de la grille                                                 & m \\  \cline{2-3}
        & Afficher le score                                                       & M \\  \cline{2-3}
        & Afficher le menu de jeu                                                     & M \\
  \hline
\end{tabular}
\normalsize

* M = Majeure / m = mineure



\subsection{Interface avec l'utilisateur}

\subsubsection{Interface Homme-Machine}

Le jeu est une application pour ordinateur qui se lance dans une fenêtre. Le joueur regarde l'écran et interagit à la fois avec son clavier et sa souris.


\subsubsection{Les menus}


\paragraph{Le menu principal}

Après le lancement de l'application, le joueur se trouve dans le menu principal où figure~:

\begin{itemize}
   \item \textbf{Lancement d'une nouvelle partie}~;
   \item \textbf{Charger une partie : }Ce menu permet de retrouver une partie à l'endroit où on l'avait laissé. Il permet aussi la suppression d'une sauvegarde~;
   \item \textbf{Le "Hall of Fame" : }On y trouve les meilleurs scores de tous les joueurs ayant utilisé cette application. Les 10 meilleurs scores et leurs statistiques associées sont visibles dans trois sous-menus différents pour chaque difficulté (novice, moyen, expert). Il y a trois statistiques représentant une partie~: le nombre d'ennemis tués, la distance parcourue, (une case = un mètre), l'or total accumulé et le temps de jeu. Le score associé à une partie est synthétisé à partir de l'ensemble de ces statistiques~;
   \item \textbf{Langue : }Permet de changer la langue parmi celles disponibles~;
   \item \textbf{Aide : }Donne accès à une base d'aide permettant à l’utilisateur de prendre facilement le jeu en main~;
   \item \textbf{Quitter le jeu}.
\end{itemize}


\newpage %pas bien

\subsubsection{Interface Graphique}

Voici un schéma général de l'interface graphique dont diposera le jeu :

%\begin{figure}[!h]
%\centering
%\includegraphics[width=12cm]{IHM.png}
%\caption{Schéma de l'IHM envisagée}
%\end{figure}

%\vspace{1cm}

%\begin{figure}[!h]
%\centering
%\includegraphics[width=12cm]{IHM2.png}
%\caption{Zoom sur le panel de contrôle et les barres} 
%\end{figure}

%----------------------------------------------------------------------------------------
%	Déroulement du projet
%----------------------------------------------------------------------------------------

\newpage %pas bien

\section{Déroulement du projet}


\subsection{Livrables}

Les livrables prévus sont :

\begin{itemize}
   \item Le présent cahier des charges validé par le client~;
   \item Dossier d'analyse~;
   \item Manuel utilisateur~;
   \item Le jeu fonctionnel (livraison le jeudi 16 mai 2014).
\end{itemize}


\subsection{Planning}

Afin de mener à bien ce projet, il est mis à disposition des étudiants 16 séances de 3h (en plus du temps libre) pour permettre aux membres de l'équipe de se retrouver et au chef d'équipe de distribuer les tâches en respectant au mieux le planning mis en place au début du projet.



\subsection{Équipe}

Les membres de l'équipe sont Rémi TREMBLAIN (chef d'équipe), Erwan MARCAHND, Colas PICARD, Kévin CROUILLERE et Anice KHOMANY.





\subsection{Outils de développement}

Voici les différents outils utilisés pour le projet :

\begin{itemize}
   \item Le language de programmation Ruby~;
   \item GTK pour Ruby pour le développement des interfaces graphiques~;
   \item YAML : outil qui permet la sérialisation de données, et qui sera donc utilisé pour la sauvegarde de données, telles que l'inventaire, la carte, la partie en cours.
\end{itemize}


%----------------------------------------------------------------------------------------
%	Gloassaire
%----------------------------------------------------------------------------------------

\section{Glossaire}

\textbf{PNJ : }Personnage Non Joueur\\
\textbf{GTK : }Gimp ToolKit\\
\textbf{POO : }Programmation Orientée Objet\\
\textbf{YAML : }YAML ain't markup language (acronyme récursif)\\


\renewcommand{\thefootnote}{\*} %Ne pas numéroter la note de bas de page
\footnotetext{Note : Les chiffres et pourcentages sont fournis à titre indicatif. Il pourront être modifiés pour une meilleure jouabilité.}



\end{document}
