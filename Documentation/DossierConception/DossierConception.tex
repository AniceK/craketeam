%%%%%%%%%%%%%%%%%%%%%%%%%%%%%%%%%%%%%%%%%
% Programming/Coding Assignment
% LaTeX Template
%
% This template has been downloaded from:
% http://www.latextemplates.com
%
% Original author:
% Ted Pavlic (http://www.tedpavlic.com)
%
% Note:
% The \lipsum[#] commands throughout this template generate dummy text
% to fill the template out. These commands should all be removed when 
% writing assignment content.
%
% This template uses a Perl script as an example snippet of code, most other
% languages are also usable. Configure them in the "CODE INCLUSION 
% CONFIGURATION" section.
%
%%%%%%%%%%%%%%%%%%%%%%%%%%%%%%%%%%%%%%%%%

%----------------------------------------------------------------------------------------
%	PACKAGES AND OTHER DOCUMENT CONFIGURATIONS
%----------------------------------------------------------------------------------------

\documentclass[11pt]{article}

\usepackage{fancyhdr} % Required for custom headers
\usepackage{lastpage} % Required to determine the last page for the footer
\usepackage{extramarks} % Required for headers and footers
\usepackage[usenames,dvipsnames]{color} % Required for custom colors
\usepackage{graphicx} % Required to insert images
\usepackage{listings} % Required for insertion of code
\usepackage{courier} % Required for the courier font
%\usepackage{lipsum} % Used for inserting dummy 'Lorem ipsum' text into the template
%\usepackage{multirow} %Ligne multiple pour les tableaux
\usepackage[utf8]{inputenc}
\usepackage{indentfirst} %Indentation début de paragraphe
\usepackage{float}


\usepackage{colortbl} %Clouleur tableau protoypes de fonctions


% Margins
\topmargin=-0.45in
\evensidemargin=0in
\oddsidemargin=0in
\textwidth=6.5in
\textheight=9.0in
\headsep=0.25in

\linespread{1.1} % Line spacing

% Set up the header and footer
\pagestyle{fancy}
\lhead{\hmwkAuthorName} % Top left header
\chead{\hmwkClass\ - \hmwkTitle} % Top center head
\rhead{\firstxmark} % Top right header
\lfoot{\lastxmark} % Bottom left footer
\cfoot{} % Bottom center footer
\rfoot{Page\ \thepage\ sur\ \protect\pageref{LastPage}} % Bottom right footer
\renewcommand\headrulewidth{0.4pt} % Size of the header rule
\renewcommand\footrulewidth{0.4pt} % Size of the footer rule

\setlength\parindent{10pt} % Removes all indentation from paragraphs




%----------------------------------------------------------------------------------------
%	NAME AND CLASS SECTION
%----------------------------------------------------------------------------------------

\newcommand{\hmwkTitle}{PROJET JEU} % Titre du document
\newcommand{\hmwkDueDate}{Mercredi 5 mars 2014} % Date
\newcommand{\hmwkClass}{DOSSIER DE CONCEPTION } % Type de document
\newcommand{\hmwkClassInstructor}{ } % Teacher/lecturer
\newcommand{\hmwkAuthorName}{Groupe A} % Your name
\newcommand{\hmwkAuthorClasse}{L3 info SPI} % Classe


%----------------------------------------------------------------------------------------
%	TITLE PAGE
%----------------------------------------------------------------------------------------

\title{
\pagenumbering{roman} \setcounter{page}{0} %La page courante sera numérotée en roman et aura l'indice 0 => Pas de numéro car pas de 0 en roman
\vspace{2in}
\textmd{\textbf{\hmwkClass:\ \hmwkTitle}}\\
\normalsize\vspace{0.1in}\small{\hmwkDueDate}\\
\vspace{0.1in}\large{\textit{\hmwkClassInstructor\ }}
\vspace{3in}
}

\author{\textbf{\hmwkAuthorName}}


\date{\hmwkAuthorClasse} % Insert date here if you want it to appear below your name

%----------------------------------------------------------------------------------------

\begin{document}

\thispagestyle{empty}
\maketitle
\newpage


%----------------------------------------------------------------------------------------
%	TABLE OF CONTENTS
%----------------------------------------------------------------------------------------

\thispagestyle{empty}
\pagenumbering{arabic} \setcounter{page}{0} %Le reste du document est numéroté en arabic à partir de la page 1
\renewcommand\contentsname{Sommaire}
\tableofcontents
\newpage


%----------------------------------------------------------------------------------------
%	Présentation général
%----------------------------------------------------------------------------------------

\newpage

\section{Présentation général}

\subsection{Objectif du document}

Partie d'Erwan

\subsection{Objectif du projet}

Partie d'Erwan

\subsection{Documents de références}

Partie d'Erwan



%----------------------------------------------------------------------------------------
%	Modélisation du projet
%----------------------------------------------------------------------------------------

\newpage


\section{Modélisation du projet}


\subsection{Les cas d'utilisations}

Partie de Kévin

\subsection{Contraintes de conception}

Partie de Kévin

\subsection{Le scénario du système}

Partie de Kévin

\subsection{Schéma de navigation}

Partie de Kévin


%----------------------------------------------------------------------------------------
%	Modélisation de la base de données
%----------------------------------------------------------------------------------------

\newpage

\section{Modélisation de la base de données}

\subsection{Dictionnaire de données}

Partie de Colas

\subsection{Modélisation conceptuel de données (MCD)}

Partie de Colas

\subsection{Script de création des tables}


%----------------------------------------------------------------------------------------
%	IHM
%----------------------------------------------------------------------------------------

\newpage %pas bien

\section{Interface Homme-Machine}

Partie d'Anice

%----------------------------------------------------------------------------------------
%	Outlis, normes et standards de programmation
%----------------------------------------------------------------------------------------

\newpage

\section{Outils, normes et standards de programmation}

\subsection{Languages utilisées}

Partie de Rémi

\subsection{Envrionnement et outils de développement}

Partie de Rémi

\subsection{Standards de programmation}

Partie de Rémi


%----------------------------------------------------------------------------------------
%	Fin du document
%----------------------------------------------------------------------------------------

\newpage

qqch à mettre ici ?

\end{document}
