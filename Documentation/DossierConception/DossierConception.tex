%%%%%%%%%%%%%%%%%%%%%%%%%%%%%%%%%%%%%%%%%
% Dossier de conception du projet Picross de l'équipe Craketeam
% 
% Version 2.0
%
% Pensez à éditer le numéro de version (décimal pour les modifications mineures, entier pour les réunions de groupes) et à iniquer vos modifications en dessous.
%
%   Version 2.0(Colas) : Modification de l'entête, des commentaires (traduits ou supprimés), et des packages.
%%%%%%%%%%%%%%%%%%%%%%%%%%%%%%%%%%%%%%%%%

%----------------------------------------------------------------------------------------
%	Packages et documention du document
%----------------------------------------------------------------------------------------

\documentclass[11pt]{article}

\usepackage{fancyhdr} % Required for custom headers
\usepackage{lastpage} % Required to determine the last page for the footer
\usepackage{extramarks} % Required for headers and footers
\usepackage[usenames,dvipsnames]{color} % Required for custom colors
\usepackage{graphicx} % Required to insert images
\usepackage{listings} % Required for insertion of code
\usepackage{courier} % Required for the courier font
\usepackage[utf8]{inputenc}
\usepackage{indentfirst} %Indentation début de paragraphe
\usepackage{float}
\usepackage{colortbl} %Clouleur tableau protoypes de fonctions


% Marges
\topmargin=-0.45in
\evensidemargin=0in
\oddsidemargin=0in
\textwidth=6.5in
\textheight=9.0in
\headsep=0.25in

\linespread{1.1} % Line spacing

% Réglages des pieds de page et des en-têtes
\pagestyle{fancy}
\lhead{\hmwkAuthorName} % Top left header
\chead{\hmwkClass\ - \hmwkTitle} % Top center head
\rhead{\firstxmark} % Top right header
\lfoot{\lastxmark} % Bottom left footer
\cfoot{} % Bottom center footer
\rfoot{Page\ \thepage\ sur\ \protect\pageref{LastPage}} % Bottom right footer
\renewcommand\headrulewidth{0.4pt} % Size of the header rule
\renewcommand\footrulewidth{0.4pt} % Size of the footer rule

\setlength\parindent{10pt} % Removes all indentation from paragraphs


\newcommand{\hmwkTitle}{PROJET JEU} % Titre du document
\newcommand{\hmwkDueDate}{Mercredi 5 mars 2014} % Date
\newcommand{\hmwkClass}{DOSSIER DE CONCEPTION } % Type de document
\newcommand{\hmwkClassInstructor}{ } % Teacher/lecturer
\newcommand{\hmwkAuthorName}{Groupe A} % Your name
\newcommand{\hmwkAuthorClasse}{L3 info SPI} % Classe


%----------------------------------------------------------------------------------------
%	Page de Titre
%----------------------------------------------------------------------------------------

\title{
\pagenumbering{roman} \setcounter{page}{0} %La page courante sera numérotée en roman et aura l'indice 0 => Pas de numéro car pas de 0 en roman
\vspace{2in}
\textmd{\textbf{\hmwkClass:\ \hmwkTitle}}\\
\normalsize\vspace{0.1in}\small{\hmwkDueDate}\\
\vspace{0.1in}\large{\textit{\hmwkClassInstructor\ }}
\vspace{3in}
}

\author{\textbf{\hmwkAuthorName}}


\date{\hmwkAuthorClasse} % Insert date here if you want it to appear below your name

%----------------------------------------------------------------------------------------

\begin{document}

\thispagestyle{empty}
\maketitle
\newpage


%----------------------------------------------------------------------------------------
%	Table de matières
%----------------------------------------------------------------------------------------

\thispagestyle{empty}
\pagenumbering{arabic} \setcounter{page}{0} %Le reste du document est numéroté en arabic à partir de la page 1
\renewcommand\contentsname{Sommaire}
\tableofcontents
\newpage


%----------------------------------------------------------------------------------------
%	Présentation générale
%----------------------------------------------------------------------------------------

\newpage

\section{Présentation général}

\subsection{Objectif du document}

Partie d'Erwan
% À EDITER!!!
\subsection{Objectif du projet}

Partie d'Erwan
% A ÉDITER!
\subsection{Documents de références}

Partie d'Erwan
% À ÉDITER!!


%----------------------------------------------------------------------------------------
%	Modélisation du projet
%----------------------------------------------------------------------------------------

\newpage


\section{Modélisation du projet}


\subsection{Les cas d'utilisations}

Partie de Kévin
% À ÉDITER!!
\subsection{Contraintes de conception}

Partie de Kévin
% À ÉDITER!!
\subsection{Le scénario du système}

Partie de Kévin
% À ÉDITER
\subsection{Schéma de navigation}

Partie de Kévin
% À ÉDITER!!

%----------------------------------------------------------------------------------------
%	Modélisation de la base de données
%----------------------------------------------------------------------------------------

\newpage

\section{Modélisation de la base de données}

\subsection{Dictionnaire de données}

Partie de Colas

\subsection{Modélisation conceptuel de données (MCD)}

Partie de Colas

\subsection{Script de création des tables}


%----------------------------------------------------------------------------------------
%	IHM
%----------------------------------------------------------------------------------------

\newpage %pas bien

\section{Interface Homme-Machine}

Partie d'Anice
%À ÉDITER
%----------------------------------------------------------------------------------------
%	Outlis, normes et standards de programmation
%----------------------------------------------------------------------------------------

\newpage

\section{Outils, normes et standards de programmation}

\subsection{Languages utilisées}

Ci dessous nous référençons la liste des différents langages utilisés dans le projet : 

\begin{itemize}
	\item Ruby : Langage de programmation orienté objet, utilisé en version 2.0.0 sur nos machines personnelles, avec une compatibilitée sur les versions antétrieurs,
	\item GTK : Bibliothèque graphique (The \textbf{G}IMP \textbf{T}ool\textbf{k}it) utilisé avec Ruby.
\end{itemize}

\subsection{Envrionnement et outils de développement}

Le développement sera effectué sur des ordinateurs suffisamment puissants auxquelles nous avons accès à l'Université du Maine, ainsi que sur nos machines personnelles.

\begin{itemize}
		\item Le système de gestion de base de données (SGBD) utilisé sera MySQL. C'est un système de gestion de bases de données relationnelles.
		\item Version utilisé de Ruby/GTK
\end{itemize}

\subsection{Standards de programmation}

Durant le développement, nous respecterons les conventions de codages ci contre : 

\begin{itemize}
	\item Les commentaires : ils doivent être rédigés en français et permettent d'expliquer de façon claire le code,
	\item Le codage : à voir les standards de programmation
	
	\item La documentation : Ruby doc 
\end{itemize}

%----------------------------------------------------------------------------------------
%	Fin du document
%----------------------------------------------------------------------------------------

\newpage

qqch à mettre ici ?

\end{document}
