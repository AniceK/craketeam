%%%%%%%%%%%%%%%%%%%%%%%%%%%%%%%%%%%%%%%%%
% Dossier de conception du projet Picross de l'équipe Craketeam
% 
% Version 2.03
%
% Pensez à éditer le numéro de version (décimal pour les modifications mineures, entier pour les réunions de groupes) et à iniquer vos modifications en dessous.
%
%   Version 2.0(Colas) : Modification de l'entête, des commentaires (traduits ou supprimés), et des packages.
%   Version 2.01(Colas) : Ajout du dictionnaire des données (non exhaustif ni terminé)
%   Version 2.02(Colas) : Modification des noms de mes parties, et complétion de ces dernières.
%   Version 2.03(Colas) : Ajout de la description des Parties Sauvegardée, et modification de la description des scores
%%%%%%%%%%%%%%%%%%%%%%%%%%%%%%%%%%%%%%%%%

%----------------------------------------------------------------------------------------
%	Packages et documention du document
%----------------------------------------------------------------------------------------

\documentclass[11pt]{article}

\usepackage{fancyhdr} % Required for custom headers
\usepackage{lastpage} % Required to determine the last page for the footer
\usepackage{extramarks} % Required for headers and footers
\usepackage[usenames,dvipsnames]{color} % Required for custom colors
\usepackage{graphicx} % Required to insert images
\usepackage{listings} % Required for insertion of code
\usepackage{courier} % Required for the courier font
\usepackage[utf8]{inputenc}
\usepackage{indentfirst} %Indentation début de paragraphe
\usepackage{float}
\usepackage{colortbl} %Clouleur tableau protoypes de fonctions


% Marges
\topmargin=-0.45in
\evensidemargin=0in
\oddsidemargin=0in
\textwidth=6.5in
\textheight=9.0in
\headsep=0.25in

\linespread{1.1} % Line spacing

% Réglages des pieds de page et des en-têtes
\pagestyle{fancy}
\lhead{\hmwkAuthorName} % Top left header
\chead{\hmwkClass\ - \hmwkTitle} % Top center head
\rhead{\firstxmark} % Top right header
\lfoot{\lastxmark} % Bottom left footer
\cfoot{} % Bottom center footer
\rfoot{Page\ \thepage\ sur\ \protect\pageref{LastPage}} % Bottom right footer
\renewcommand\headrulewidth{0.4pt} % Size of the header rule
\renewcommand\footrulewidth{0.4pt} % Size of the footer rule

\setlength\parindent{10pt} % Removes all indentation from paragraphs


\newcommand{\hmwkTitle}{PROJET JEU} % Titre du document
\newcommand{\hmwkDueDate}{Mercredi 5 mars 2014} % Date
\newcommand{\hmwkClass}{DOSSIER DE CONCEPTION } % Type de document
\newcommand{\hmwkClassInstructor}{ } % Teacher/lecturer
\newcommand{\hmwkAuthorName}{Groupe A} % Your name
\newcommand{\hmwkAuthorClasse}{L3 info SPI} % Classe


%----------------------------------------------------------------------------------------
%	Page de Titre
%----------------------------------------------------------------------------------------

\title{
\pagenumbering{roman} \setcounter{page}{0} %La page courante sera numérotée en roman et aura l'indice 0 => Pas de numéro car pas de 0 en roman
\vspace{2in}
\textmd{\textbf{\hmwkClass:\ \hmwkTitle}}\\
\normalsize\vspace{0.1in}\small{\hmwkDueDate}\\
\vspace{0.1in}\large{\textit{\hmwkClassInstructor\ }}
\vspace{3in}
}

\author{\textbf{\hmwkAuthorName}}


\date{\hmwkAuthorClasse} % Insert date here if you want it to appear below your name

%----------------------------------------------------------------------------------------

\begin{document}

\thispagestyle{empty}
\maketitle
\newpage


%----------------------------------------------------------------------------------------
%	Table de matières
%----------------------------------------------------------------------------------------

\thispagestyle{empty}
\pagenumbering{arabic} \setcounter{page}{0} %Le reste du document est numéroté en arabic à partir de la page 1
\renewcommand\contentsname{Sommaire}
\tableofcontents
\newpage


%----------------------------------------------------------------------------------------
%	Présentation générale
%----------------------------------------------------------------------------------------

\newpage

\section{Présentation général}

\subsection{Objectif du document}

Ce document est rédigé dans le cadre du projet de troisième année de licence SPI option informatique de l'Université du
Maine. Le Dossier de Conception vient en complément du Cahier des Charges. Il a pour objectif de monter la modélisation
du système à partir des besoins et contraintes exprimés dans le Cahier des Charges. Il informe également sur les outils
et les normes utilisés pour la conception du produit.

\subsection{Objectif du projet}

L'objectif de ce projet est de concevoir et de développer une application de Picross, ainsi que de rédiger les documents
inhérant à la gestion et la finalisation d'un projet informatique. Les utilisateurs doivent pouvoir, bien évidemment,
jouer, mais il doivent également pouvoir créer des grilles, modifier la taille des grilles, et charger les partie
préalablement sauvegarder. Il doit également être possible de jouer sous son propre profil afin d'avoir accès aux
options les plus poussées de l'application.

\subsection{Documents de références}

Afin de réaliser le présent Dossier de Conception nous nous sommes appuyés sur le Cahier des Charges ainsi que sur les
exemples que nous avons récupérés grâce à un ancien élève de DUT informatique.



%----------------------------------------------------------------------------------------
%	Modélisation du projet
%----------------------------------------------------------------------------------------

\newpage


\section{Modélisation du projet}


\subsection{Les cas d'utilisations}

Partie de Kévin
% À ÉDITER!!
\subsection{Contraintes de conception}

Partie de Kévin
% À ÉDITER!!
\subsection{Le scénario du système}

Partie de Kévin
% À ÉDITER
\subsection{Schéma de navigation}

Partie de Kévin
% À ÉDITER!!

%----------------------------------------------------------------------------------------
%	Modélisation de la base de données
%----------------------------------------------------------------------------------------

\newpage

\section{Modélisation de la Sauvegarde des données}

\subsection{Dictionnaire de données}


\begin{tabular}{|c|c|c|l|}\hline
    {\bf Nom de l'entité} & {\bf Type } & {\bf Code} & {\bf Commentaires}\\\hline
    Grille & Class Ruby Grille & & une collection de case, jointe à un nom et des paramètres ("ALEA20", 20/11/2014, ----)\\\hline
    Profil & Chaîne de caractère & & le profil joueur, constitué uniquement du nom du profil ("joueur1")\\\hline
    Scores & Tableau & & un tableau à trois colonnes, comprenant le nom de la grille, le profil et le score\\\hline
    Parties Sauvegardées & Class Ruby Partie & & Partie en cours, sauvegardée avec tous ses paramètres (profil = "", grille ="", temps=--:--, ....) \\\hline 
\end{tabular}

\subsection{Description des données sauvegardées et de leur stockage}
L'ensemble des données sauvegardées, présentées brièvement ci-dessus, est ici décrite plus en détail.

\begin{description}
    \item [Grille] : les grilles sont les supports du jeu( cf Cahier des Charges 3.1.1). Pour permettre à plusieurs utilisateurs de jouer sur une même grille, il est nécessaire de sauvegarder ces objets. Les grilles sont constituées d'un collection de case, d'un nom, attribué à la création et de divers paramètres, stockés sous formes d'entiers. Elles sont stockés dans un même fichier, les unes après les autres, à leur création, qu'elle soit manuelle ou aléatoire. Ce fichier comporte trois grilles par défaut.
        % Le stockage peut également se faire autrement, surtout si nous souhaitons proposer aux joueurs de jouer à leurs propres grilles de préférence. Cela peut donc se faire dans un tableau à deux colonnes, l'une indiquant le profil créateur, l'autre contenant la grille. Un tri permettrai de faire ressortir les grilles du profil pour les lui proposer, avec une option "voir toutes les grilles" qui proposerai toutes les grilles. Cependant, on peut envisager que la grille comprend dans ses paramètres le profil du créateur. Le stockage alors n'aurait pas à être modifier, uniquement le comportement lors du chargement du fichier, avec l'enregistrement en mémoire vive d'un tableau pour lequel on récupèrerai le profil du créateur.
    \item [Profil] : Les profils sont les noms que choisissent les utilisateurs(cf Cahier des Charges 3.1.2, qui les définissent dans le jeu, et qui les lient aux grilles créées, aux parties jouées et aux scores. Un profil est constitué d'une simple chaîne de caractères, choisie par l'utilisateur à la création du profil. Cette chaîne est sérialisée à la création, dans un tableau contenant l'ensemble des profils créés, le tout stocké sur un même fichier spécifique aux profils. Ce fichier comporte un profil "par défaut".
    \item [Scores] : Les scores sont des entiers, dépendant du temps réalisé pour finir la Grille. La donnée à sauvegardée "Scores" est un tableau à trois colonne, comprenant le score lui même, le nom de la grille sur lequel il a été réalisé, et le nom du profil l'ayant réalisé. Ce tableau est stocké sur un fichier propre.
        % Peut-être faudra il prévoir plusieurs tableaux, en fonction des différents modes (une colonne indiquant le nombre grille d'affilée en Mode Aventure, etc).
    \item [Parties Sauvegardées] : Elles correspondent à la sérialisation d'une instance de la classe Partie, contenant un certin nombre de paramètres et d'informations, comme la grille utilisée, l'utilisateur, le temps écoulé, etc. Ces paramètres sont conservés sur un fichier lors de la sérialisation. L'utilisateur peut ainsi quitter une partie, fermer le programme, et reprendre cette partie plus tard. Chaque Partie Sauvegardée est conservée sur un fichier propre.
\end{description}

\subsection{Méchanismes de Sauvegarde des Données}


%----------------------------------------------------------------------------------------
%	IHM
%----------------------------------------------------------------------------------------

\newpage %pas bien

\section{Interface Homme-Machine}

Partie d'Anice
%À ÉDITER
%----------------------------------------------------------------------------------------
%	Outlis, normes et standards de programmation
%----------------------------------------------------------------------------------------

\newpage

\section{Outils, normes et standards de programmation}

\subsection{Languages utilisées}

Ci dessous nous référençons la liste des différents langages utilisés dans le projet : 

\begin{itemize}
	\item Ruby : Langage de programmation orienté objet, utilisé en version 2.0.0 sur nos machines personnelles, avec une compatibilitée sur les versions antétrieurs,
\end{itemize}

\subsection{Envrionnement et outils de développement}

Le développement sera effectué sur des ordinateurs suffisamment puissants auxquelles nous avons accès à l'Université du Maine, ainsi que sur nos machines personnelles.

\begin{itemize}
		\item Version utilisé de Ruby : 2.0,
		\item GTK : Bibliothèque graphique (The \textbf{G}IMP \textbf{T}ool\textbf{k}it) utilisé avec Ruby.
\end{itemize}

\subsection{Standards de programmation}

Durant le développement, nous respecterons les conventions de codages ci contre : 

\begin{itemize}
	\item Les commentaires doivent être rédigés en français et permettent d'expliquer de façon claire le code.
    \item Le code doit être rédigé de la manière la plus lisible possible (indenté et explicite).
	\item La documentation du code sera faites à l'aide de Ruby doc.
\end{itemize}

\end{document}
